\begin{abstract}
Understanding emotions and expressions is a task of interest across multiple disciplines, especially for improving user experiences. Contrary to the common perception, it has been shown that emotions are not discrete entities but instead exist along a continuum. People understand discrete emotions differently due to a variety of factors, including cultural background, individual experiences, and cognitive biases. Therefore, most approaches to expression understanding, particularly those relying on discrete categories, are inherently biased. In this paper, we present a comparative in-depth analysis of two common datasets (\affectnet{}  and \emotic{}) equipped with the components of the circumplex model of affect. Further, we propose a model for the prediction of facial expressions tailored for lightweight applications. Using a small-scaled MaxViT-based model architecture, we evaluate the impact of discrete expression category labels % (\textit{Neutral, Happiness, Sadness, Surprise, Fear, Disgust, Anger, Contempt}) 
in training with the continuous \va{} labels. We show that considering valence and arousal in addition to discrete category labels helps to significantly improve expression inference.  The proposed model outperforms the current state-of-the-art models on \affectnet{}, establishing it as the best-performing model for inferring \va{} achieving a 7\% lower RMSE. Training scripts and trained weights to reproduce our results can be found here: \url{https://github.com/wagner-niklas/CAGE_expression_inference}.
\end{abstract}